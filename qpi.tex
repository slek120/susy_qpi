\documentclass[11pt, oneside]{article} 
\usepackage{amsmath}
\begin{document}

\section{Bogoliubov Tranformation for the Kondo Lattice}

We start from the Hamiltonian for the Kondo lattice

$$ H = - \sum_{i,\sigma} t_{ij}c_{i\sigma}^\dagger c_{j\sigma} + 
J \sum_r \mathbf{S}_r\cdot \mathbf{s}_r +
\sum_{\langle r, s \rangle} I_{rs} \mathbf{S}_r \cdot \mathbf{S}_s $$

where $\mathbf{S}_{i},\mathbf{s}_{i}$ are the spin operators for the
localized $f$-electrons and the itinerant $d$-electrons.
Next, we use that

$$ \mathbf{S}_{r}=\mathbf{S}_{r}^{f}+\mathbf{S}_{r}^{b} $$

representing the fermionic and bosonic components of the spin. For the
fermionic component, we use

$$ \mathbf{S}_{r}^{f} =\frac{1}{2}\sum_{\alpha ,\beta }f_{r,\alpha }^{\dagger
}\mathbf{\sigma }_{\alpha \beta }f_{r,\beta } $$

$$ \mathbf{s}_{r} =\frac{1}{2}\sum_{\alpha ,\beta }c_{r,\alpha }^{\dagger }
\mathbf{\sigma }_{\alpha \beta }c_{r,\beta } $$

We will assume that the bosonic part of the spins orders
antiferromagnetically, and hence use the Holstein-Primakoff transformation
on the $A$ sublattice

$$ S_{r}^{z} = S-a_{r}^{\dagger }a_{r} $$
$$ S_{r}^{+} = \sqrt{2S}\sqrt{1-\frac{a_{r}^{\dagger }a_{r}}{2S}}a_{r} $$
$$ S_{r}^{-} = \sqrt{2S}a_{r}^{\dagger }\sqrt{1-\frac{a_{r}^{\dagger }a_{r}}{2S}} $$

and on the $B$ sublattice

$$ S_{r}^{z} = -S+a_{r}^{\dagger }a_{r} $$
$$ S_{r}^{-} = \sqrt{2S}a_{r}^{\dagger }\sqrt{1-\frac{a_{r}^{\dagger }a_{r}}{2S}} $$
$$ S_{r}^{+} = \sqrt{2S}\sqrt{1-\frac{a_{r}^{\dagger }a_{r}}{2S}}a_{r} $$



We obtain

$$ H_{MF}=\sum_{k}
\left(\begin{array}{cccc}
c_{k,\sigma }^{\dagger } & f_{k,\sigma }^{\dagger } & c_{k+Q,\sigma
}^{\dagger } & f_{k+Q,\sigma }^{\dagger }
\end{array} \right) 
\left( \begin{array}{cccc}
\varepsilon _{k} & -V & U_{c} & 0 \\
-V & \chi _{k} & 0 & U_{f} \\
U_{c} & 0 & \varepsilon _{k+Q} & -V \\
0 & U_{f} & -V & \chi _{k+Q}
\end{array}\right)
\left(\begin{array}{c}
c_{k,\sigma } \\
f_{k,\sigma } \\
c_{k+Q,\sigma } \\
f_{k+Q,\sigma }
\end{array}\right) $$

$$ \hat{H_k} = 
\left( \begin{array}{cccc}
\varepsilon _{k} & -V & U_{c} & 0 \\
-V & \chi _{k} & 0 & U_{f} \\
U_{c} & 0 & \varepsilon _{k+Q} & -V \\
0 & U_{f} & -V & \chi _{k+Q}
\end{array}\right) $$

Where

\begin{eqnarray*}
\varepsilon_k &=& -2 \cdot t (\cos(k_x) + \cos(k_y)) - \mu \\
\chi_k &=& -2 \cdot \chi_0 (\cos(k_x) + \cos(k_y)) - \varepsilon_f \\
V &=& \frac{J N}{4} \sum_{q,\alpha} \langle f_{q\alpha}^\dagger c_{q\alpha} \rangle \\
U_c &=& \frac{J S}{2} sgn(\sigma) \\
U_f &=& \frac{I S}{2} sgn(\sigma)
\end{eqnarray*}

Then the Green's function matrix is defined as
$$ \hat{G_R}(k,\omega_n+i\delta) = [(\omega_n+i\delta)\hat{I} - \hat{H_k}]^{-1} $$
From which we calculate the QPI spectrum
$$ g(q,\omega)=\int \frac{d^2x}{(2 \pi)^2} [\hat{G_R}(k,\omega)]_{11}[\hat{G_R}(k+q,\omega)]_{11} $$

\end{document}  